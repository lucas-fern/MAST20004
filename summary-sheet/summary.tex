\documentclass[titlepage,twocolumn]{article}

\usepackage[a4paper,top=2cm,bottom=2cm,left=1cm,right=1cm]{geometry}
\usepackage[fleqn]{amsmath}
\usepackage{amssymb}
\usepackage{fancyhdr}
\usepackage{hyperref}
\usepackage{graphicx}
\usepackage{algorithm2e}
\usepackage{nopageno}
\usepackage{mathtools}

\title{MAST20004 Probability - Summary Notes}
\date{\today}
\author{Lucas Fern (1080613)}
\lhead{MAST20004 Probability - Summary Notes}
\rhead{Lucas Fern (1080613)}

\pagestyle{fancy}

\begin{document}
\section*{Distributions}
\subsection*{Binomial}
If $X$ is the number of successes in $n$ independent Bernoulli trials with probability of success $p$ then\\$X\stackrel{d}{=}\mbox{Bi}(n,p)$.
\begin{align*}
    &p_X(x)={\binom{n}{x}}p^x(1-p)^{n-x}\\
    &\mathbb{E}(X)=np\\
    &\mbox{Var}(X)=np(1-p)
\end{align*}
$X$ has recursive formula: $$\frac{p_X(x)}{p_X(x-1)}=\frac{\frac{n+1}{x}-1}{\frac{1}{p}-1}$$

\subsection*{Geometric}
If $N$ is the number of failures before a success in a sequence of independent Bernoulli trials with probability of success $p$ then $N\stackrel{d}{=}\mbox{G}(p)$.
\begin{align*}
    &p_N(n)=(1-p)^{n}p\\
    &\mathbb{E}(N)=\frac{1-p}{p}\\
    &\mbox{Var}(N)=\frac{1-p}{p^2}
\end{align*}
The geometric distribution has the memoryless property.

\subsection*{Negative Binomial}
If $Z$ is the number of failures before the $r^{th}$ success in a sequence of independent Bernoulli trials with probability of success $p$ then $Z\stackrel{d}{=}\mbox{Nb}(r,p)$.
\begin{align*}
    p_Z(z)&={\binom{z+r-1}{r-1}}p^r(1-p)^z\\
    &={\binom{-r}{z}}p^r(p-1)^z
\end{align*}
Where ${\binom{x}{k}} = \frac{x(x-1)(x-2)...(x-k+2)(x-k+1)}{k!}$ for $x \in \mathbb{R}$

\subsection*{Hypergeometric}
When sampling $n$ items from a population of $N$ without replacement, where $D$ are defective, the amount of defective items selected is $X$ and; $X\stackrel{d}{=}\mbox{Hg}(n,D,N)$.
\begin{align*}
    &p_X(x)=\frac{{\binom{D}{x}}{\binom{N-D}{n-x}}}{{\binom{N}{n}}}\\
    &\mathbb{E}(X)=\frac{nD}{N}\\
    &\mbox{Var}(X)=\frac{nD(N-D)}{N^2}\cdot(1-\frac{n-1}{N-1})
\end{align*}
\vfill
\subsection*{Poisson}
An analogue for the Binomial distribution in continuous time. If $\alpha$ is the expected amount of successes over unit time (the Poisson rate), then the amount of successes $X\stackrel{d}{=}\mbox{Pn}(\alpha)$ and the amount of successes in $t$ units of time $\stackrel{d}{=}\mbox{Pn}(\alpha t)$.
\begin{align*}
    &p_X(x)=\frac{e^{-\alpha}(\alpha)^x}{x!}\\
    &\mathbb{E}(X)=\mbox{Var}(X)=\alpha
\end{align*}
The Poisson distribution has the memoryless property and can be used to approximate the Binomial distribution for small $p$, ($p < 0.05$) where Bi$(n,p) \stackrel{d}{\approx}\mbox{Pn}(np)$.

\subsection*{Discrete Uniform}
A representation of discrete events with equal probabilities of all outcomes, for an $X$ that can take integer values between $m$ and $n$, $X\stackrel{d}{=}\mbox{U}(m,n)$.
\begin{align*}
    &p_X(x)=\frac{1}{n-m+1}\\
    &\mathbb{E}(X)=\frac{m+n}{2}\\
    &\mbox{Var}(X)=\frac{1}{12}((n-m+1)^2-1)
\end{align*}

\subsection*{Continuous Uniform}
A representation of continuous events with equal probabilities, for an $X$ that can take any real value between $a$ and $b$, $X\stackrel{d}{=}\mbox{R}(a,b)$.
\begin{align*}
    &f_X(x)=\frac{1}{b-a}\\
    &\mathbb{E}(X)=\frac{a+b}{2}\\
    &\mbox{Var}(X)=\frac{1}{12}(b-a)^2
\end{align*}

\subsection*{Exponential}
A continuous analogue of the geometric distribution, modelling the waiting time until an event occurs in continuous time. For an event that has probability of $\alpha$ of occurring over one period of time, the waiting time until the event occurs $T$ follows $T\stackrel{d}{=}\mbox{exp}(\alpha)$.
\begin{align*}
    &f_T(t)=\begin{cases} \alpha e^{-\alpha t}, & t \geq 0 \\ 0, & t < 0 \end{cases}\\
    &\mathbb{E}(T)=\frac{1}{\alpha}\\
    &\mbox{Var}(T)=\frac{1}{\alpha ^2}
\end{align*}
The exponential distribution has the lack of memory property such that $\mathbb{P}(T \geq x+y\ |\ T \geq x) = \mathbb{P}(T \geq y)$.
\vfill
\subsection*{Gamma}
A continuous analogue of the negative binomial distribution, modelling the waiting time until $r$ events occur in continuous time. For an event that has probability of $\alpha$ of occurring over one period of time, the waiting time until the $r^{th}$ event occurs - $T$ - follows $T\stackrel{d}{=} \gamma(r, \alpha)$.
\begin{align*}
    &f_T(t)=\frac{\alpha^r t^{r-1} e^{-\alpha t}}{\Gamma(r)},\ t > 0\\
    &F_T(t)=1-\sum_{k=0}^{r-1}\frac{(\alpha t)^k}{k!}e^{-\alpha t}\\
    &\mathbb{E}(T^k)=\frac{\Gamma(r+k)}{\Gamma(r)\alpha^k}\\
    &\mathbb{E}(T)=\frac{r}{\alpha}\\
    &\mbox{Var}(T)=\frac{r}{\alpha ^2}
\end{align*}
$$\mbox{where }\Gamma(r) = \int_0^\infty e^{-x}x^{r-1} dx = (r-1)\Gamma(r-1) \mbox{ for all } r > 0$$
$$\Gamma(1) = 1 \mbox{ and } \Gamma(k) = (k-1)! \mbox{ for any } k \in \mathbb{N}$$

\subsection*{Beta}
A random variable $X$ has a beta distribution with parameters $\alpha >0$ and $\beta > 0$ and is denoted $X\stackrel{d}{=}\mbox{Beta}(\alpha, \beta)$ if it has the following properties:
\begin{align*}
    &f_X(x)=\begin{cases} \frac{x^{\alpha-1}(1-x)^{\beta-1}}{B(\alpha,\beta)}, & 0 \leq x \leq 1 \\ 0, & \mbox{elsewhere} \end{cases}\\
    &\mathbb{E}(X^k)=\frac{\Gamma(\alpha+k)\Gamma(\alpha+\beta)}{\Gamma(\alpha)\Gamma(\alpha+\beta+k)}\\
    &\mathbb{E}(X)=\frac{\alpha}{\alpha+\beta}\\
    &\mbox{Var}(X)=\frac{\alpha \beta}{(\alpha+\beta)^2(\alpha+\beta+1)}
\end{align*}
$$\mbox{where }B(\alpha,\beta) = \frac{\Gamma(\alpha)\Gamma(\beta)}{\Gamma(\alpha+\beta)} = \int_0^1x^{\alpha-1}(1-x)^{\beta-1}dx.$$
The distribution is $\cup$ shaped for $\alpha = \beta < 1$ and $\cap$ shaped for $\alpha = \beta > 1$ and Beta$(1,1) =$ R$(0,1)$.

\subsection*{Pareto}
The Pareto distribution is shaped similarly to the exponential distribution, though the tail of an exponential distribution is thinner. A random variable $X$ has a pareto distribution denoted $X\stackrel{d}{=}\mbox{Pareto}(\alpha, \gamma)$ with parameters $\{\alpha,\gamma \} \in \mathbb{R}^+$ if it has the properties:
\begin{align*}
    &f_X(x)=\frac{\gamma \alpha^\gamma}{x^{\gamma+1}}, \mbox{ for } \alpha \leq x < \infty\\
    &\mathbb{E}(X)=\frac{\gamma \alpha}{\gamma - 1}, \mbox{ for } \gamma > 1\\
    &\mbox{Var}(X)=\frac{\gamma \alpha^2}{(\gamma-1)^2(\gamma - 2)}, \mbox{ for } \gamma > 2
\end{align*}
\vfill
\subsection*{Normal}
The Normal distribution is parameterised by its mean and variance $\mu$ and $\sigma^2$. A normally distributed random variable $X$ is written $X\stackrel{d}{=}\mbox{N}(\mu, \sigma^2)$. The case N(0,1) is the \textit{Standard Normal Distribution} and it's pdf is denoted $\varphi(z)$.
\begin{align*}
    &f_X(x)=\frac{1}{\sigma \sqrt{2\pi}}e^{-\frac{1}{2\sigma^2}(x-\mu)^2}, \mbox{ for } x \in \mathbb{R}\\
    &\mathbb{E}(X)=\mu\\
    &\mbox{Var}(X)=\sigma^2
\end{align*}
For $Z \stackrel{d}{=}\mbox{N}(0,1)$ (the Standard Normal Distribution):
\begin{align*}
    &\mathbb{E}(Z^{2k})=\frac{(2k)!}{2^k k!}\\
    &\mathbb{E}(Z^{2k+1})=0
\end{align*}
The Normal distribution can be used to approximate:
\begin{itemize}
    \item $X \stackrel{d}{=}\mbox{Bi}(n,p) \stackrel{d}{\approx}\mbox{N}(np,np(1-p))$ for large $n$ and $p$ not close to 0 or 1. ($np > 5$ and $n(1-p) > 5$)
    \item $X \stackrel{d}{=}\mbox{Pn}(\lambda) \stackrel{d}{\approx}\mbox{N}(\lambda, \lambda)$ for large $\lambda$.
    \item $X \stackrel{d}{=} \gamma(r,\alpha) \stackrel{d}{\approx}\mbox{N}(\frac{r}{\alpha}, \frac{r}{\alpha^2})$ for large $r$.
\end{itemize}

\subsection*{Weibull}
The Weibull distribution is used to model survival data, where the hazard function, or rate of failures at a specific time may increase, decrease or be constant. It is parameterised by $\beta$ and $\gamma$, written $X\stackrel{d}{=}\mbox{Weibull}(\beta, \gamma)$ and has the properties:
\begin{align*}
    &f_X(x)=\frac{\gamma x^{\gamma - 1}}{\beta^\gamma}e^{-(\frac{x}{\beta})^\gamma}, \mbox{ for } 0 \leq x < \infty\\
    &\mathbb{E}(X)=\beta \Gamma\left(\frac{\gamma + 1}{\gamma}\right)\\
    &\mbox{Var}(X)=\beta^2 \left[\Gamma\left(\frac{\gamma + 2}{\gamma}\right)-\Gamma\left(\frac{\gamma + 1}{\gamma}\right)^2\right]
\end{align*}
The hazard rate of any random variable $T$ is given by $\frac{f_T(t)}{1-F_T(t)}$, giving the Weibull distribution a hazard rate of $\gamma \frac{t^{\gamma-1}}{\beta \gamma}$.

\subsection*{Cauchy}
The Cauchy distribution has parameters $m$ - the location, and $a$ - the scale parameter, and is written $X\stackrel{d}{=}\mbox{C}(m, a)$.
\begin{align*}
    &f_X(x)=\frac{1}{\pi} \frac{a}{a^2 + (x-m)^2}, \mbox{ for } -\infty < x < \infty
\end{align*}
It does not have a defined mean or variance.

\subsection*{Lognormal}
If $X \stackrel{d}{=} \mbox{N}(\mu, \sigma^2)$ and $Y \stackrel{d}{=} e^X$, then $Y$ has a lognormal distribution with the same parameters $Y \stackrel{d}{=} \mbox{LN}(\mu, \sigma^2)$.
\begin{align*}
    &f_Y(y)=\frac{1}{\sqrt{2\pi}\sigma y}e^{-\frac{(\ln y - \mu)^2}{2\sigma^2}}, \mbox{ for } y > 0\\
    &\mathbb{E}(Y^r)=e^{r\mu + \frac{1}{2}r^2 \sigma^2},\ r \geq 0\\
    &\mbox{Var}(Y)=e^{2\mu + \sigma^2}(e^{\sigma^2}-1)
\end{align*}

\subsection*{Bivariate Normal}
A bivariate normal distribution is defined by the means and variances of the univariate normal distributions $X$ and $Y$ and their correlation coefficient $\rho$. We write $(X,Y) \stackrel{d}{=} \mbox{N}_{2}(\mu_X, \mu_Y, \sigma^2_X, \sigma^2_Y, \rho)$, or for the standard bivariate normal with $\mu_X = \mu_Y = 0$ and $\sigma^2_X = \sigma^2_Y = 1$, $\mbox{N}_{2}(\rho)$.
$$
f_{(X,Y)}(x,y)=\frac{1}{2 \pi \sqrt{1 - \rho^2}}\exp{\left(\frac{-1}{2(1-\rho^2)}(x^2 - 2 \rho x y + y^2)\right)}$$ $$
(X|Y=y) \stackrel{d}{=} \mbox{N}(\mu_X + \rho \sigma_X \frac{(y-\mu_Y)}{\sigma_Y}, \sigma_X^2(1-\rho^2))
$$
The correlation coefficient $\rho$ is calculated as $\frac{\mbox{Cov}(X,Y)}{\sigma_X \sigma_Y}$ where $\mbox{Cov}(X,Y) = \mathbb{E}((X-\mu_X)(Y-\mu_Y)) = \mathbb{E}(XY) - \mathbb{E}(X)\mathbb{E}(Y) \\ = \mathbb{E}(X \mathbb{E}(Y|X)) - \mathbb{E}(X)\mathbb{E}(Y)$.

\section*{Convolutions \& Linear Combinations}
If $X$ and $Y$ are \textbf{independent} their sum (convolution) is:
\begin{alignat*}{2}
    p_{X+Y}(z) &= \sum_{x \in S_X} p_X(x)p_Y(z-x)\\
               &= \sum_{y \in S_Y} p_X(z-y)p_Y(y), &&\ (X,Y) \mbox{ discrete,}\\
    f_{X+Y}(z) &= \int_{-\infty}^{\infty} f_X(x)f_Y(z-x)\\
               &= \int_{-\infty}^{\infty} f_X(z-y)f_Y(y), &&\ (X,Y) \mbox{ continuous.}
\end{alignat*}
And for $Z_1 = aX+bY$, $Z_2 = cX+dY$:
\begin{align*}
    &\mathbb{E}(Z_1) = a\mathbb{E}(X) + b\mathbb{E}(Y)\\
    &\mbox{Cov}(Z_1, Z_2) = ac \mbox{Var}(X) + 2ab \mbox{Cov}(X,Y) + bd \mbox{Var}(Y)\\
    &\mbox{Var}(Z_1) = a^2 \mbox{Var}(X) + 2ab \mbox{Cov}(X,Y) + b^2 \mbox{Var}(Y)
\end{align*}
where the variance property is a special case; $\mbox{Cov}(Z_1, Z_1) = \mbox{Var}(Z_1)$.

\section*{Conditioning on Random Variables}
$\mathbb{E}(X|Y) = \mathbb{E}(X|Y=y)$ is the \textit{conditional expectation} of $X$ given $Y$.\\[2mm]
$\mathbb{E}(X|Y) \coloneqq \int_{-\infty}^{\infty}x f_{X|Y}(x|y)dx$, replacing the integral for a sum over $x \in S_X$ for the discrete case.\\[2mm]
The \textbf{Law of Total Expectation} is $\mathbb{E}(\mathbb{E}(X|Y)) = \mathbb{E}(X)$, where we take the outer integral over all values of $y$ and then rearrange from $\int_{-\infty}^{\infty}(\int_{-\infty}^{\infty} x f_{X|Y}(x|y) dx) f_Y(y) dy$ to get $\int_{-\infty}^{\infty}\int_{-\infty}^{\infty} x f_{(X,Y)}(x,y) dy dx = \int_{-\infty}^{\infty} x f_X(x) dx$.\\[2mm]
Then for variance, $\mbox{Var}(X) = \mbox{Var}(\mathbb{E}(X|Y)) + \mathbb{E}(\mbox{Var}(X|Y))$.

\section*{Generating Functions}
\subsection*{Probability Generating Functions (PGF's)}
The PGF exists only for non-negative integer valued random variables and the PGF of $X$ is given by $$P_X(z) = \mathbb{E}(z^X) = \sum_{x=0}^{\infty}p_X(x)z^x.$$ We then get that $$p_X(k) = \mathbb{P}(X=k) = \frac{P_X^{(k)}(0)}{k!}$$
\subsubsection*{Properties of the PGF}
\begin{itemize}
    \item $P_X(z) = \mathbb{E}(z^X) = \mathbb{E}(\mathbb{E}(z^X|Y))$.
    \item $P_X^\prime(1) = \mathbb{E}(X)$ and $P_X^{\prime\prime}(1) = \mathbb{E}(X(X-1))$,
    \item $\implies \mbox{Var}(X) = P_X^{\prime\prime}(1) + P_X^{\prime}(1) - P_X^{\prime}(1)^2$.
    \item For independent $X$ and $Y$,\\$P_{X+Y}(z)=P_X(z)P_Y(z)$ since\\$\mathbb{E}(z^{X+Y}) = \mathbb{E}(z^X)\mathbb{E}(z^Y)$.
\end{itemize}
\paragraph{Binomial PGF}
If $X\stackrel{d}{=}\mbox{Bi}(n,p)$, then: $$P_X(z) = (1-p+pz)^n,\ \mbox{for } z \in \mathbb{R}.$$
\paragraph{Poisson PGF}
If $X\stackrel{d}{=}\mbox{Pn}(\lambda)$, then: $$P_X(z) = e^{-\lambda (1-z)},\ \mbox{for } z \in \mathbb{R}.$$
\paragraph{Negative Binomial PGF}
If $X\stackrel{d}{=}\mbox{Nb}(r,p)$, then: $$P_X(z) = p^r(1-(1-p)z)^{-r},\ \mbox{for } |z| < \frac{1}{1-p}.$$ The PGF for the \textbf{geometric} distribution can be derived from this as $X\stackrel{d}{=}\mbox{G}(p)\stackrel{d}{=}\mbox{Nb}(1,p)$

\subsection*{Moment Generating Functions (MGF's)}
The $k^{th}$ moment (about the origin) of a random variable $X$ is $\mu_k = \mathbb{E}(X^k)$.\\[2mm]
The $k^{th}$ \textit{central} moment (about the mean) of a random variable $X$ is $\nu_k = \mathbb{E}((X-\mu)^k)$.\\[2mm]
The MGF is defined over all $t \in \{t: \mathbb{E}(e^{tX}) < \infty\}$ as $$M_X(t) = \mathbb{E}(e^{tX}) = \int_{x \in S_X}e^{tx}f_X(x)dx,$$ replacing the integral with a sum in the discrete case as the MGF is defined for both.
\subsubsection*{Properties of the MGF}
\begin{itemize}
    \item $M_X(0) = 1$, $M_X^\prime(0) = \mathbb{E}(X)$ and $M_X^{\prime\prime}(0) = \mathbb{E}(X^2)$,
    \item $\implies \mbox{Var}(X) = M_X^{\prime\prime}(0) - M_X^{\prime}(0)^2$.
    \item $\mu_k = M_X^{(k)}(0)$.
    \item If $Y = aX + b$ then $M_Y(t) = e^{bt}M_X(at)$.
    \item For \textbf{independent} $X$ and $Y$, $M_{X+Y}(t) = M_{X}(t)M_{Y}(t)$.
    \item If $X$ is a discrete RV defined on the non-negative integers then $M_X(t) = P_X(e^t)$ and $P_X(z) = M_X(\log z)$.
    \item The central moment generating function\\$N_X(t) = \mathbb{E}\left(e^{(X-\mu)t}\right)$
\end{itemize}
\paragraph{Binomial MGF}
If $X\stackrel{d}{=}\mbox{Bi}(n,p)$, then: $$M_X(t) = ((1-p)+pe^t)^n$$
\paragraph{Geometric MGF}
If $X\stackrel{d}{=}\mbox{G}(p)$, then: $$M_X(t) = \frac{pe^t}{1-(1-p)e^t}$$
\paragraph{Exponential MGF}
If $X\stackrel{d}{=}\mbox{exp}(\alpha)$, then: $$M_X(t) = \frac{\alpha}{\alpha - t}$$
\paragraph{Poisson MGF}
If $X\stackrel{d}{=}\mbox{Pn}(\lambda)$, then: $$M_X(t) = e^{\lambda(e^t-1)} = 1 + \lambda t + \lambda(\lambda + 1)\frac{t^2}{2} + \dots$$
\paragraph{Gamma MGF}
If $X\stackrel{d}{=}\gamma(r,\alpha)$, then: $$M_X(t) = \left( 1- \frac{t}{\alpha} \right)^{-r} = 1 + \frac{rt}{\alpha} + \frac{r(r+1)}{\alpha^2}\frac{t^2}{2} + \dots$$
\paragraph{Normal MGF}
If $X\stackrel{d}{=}\mbox{N}(\mu, \sigma^2)$, then:
\begin{align*}
    &M_X(t) = e^{\mu t + \frac{1}{2}\sigma^2t^2},\mbox{ and}\\
    &N_X(t) = e^{\frac{1}{2}\sigma^2t^2}
\end{align*}

\subsection*{Cumulant Generating Functions (CGF's)}
The CGF of a random variable $X$ is given by $$K_X(t) = \ln M_X(t)$$ and $\kappa_r = K_X^{(r)}(0)$ is the $r^{th}$ cumulant of $X$.
\subsubsection*{Properties of the CGF}
\begin{itemize}
    \item For \textbf{independent} $X$ and $Y$, $K_{X+Y}(t)=K_{X}(t)+K_{Y}(t)$.
    \item $\kappa_1 = \mathbb{E}(X)$.
    \item $\kappa_2 = \mbox{Var}(X)$.
    \item $\kappa_3 = \mathbb{E}(X^3)$ is the \textit{skewness} of $X$.
    \item $\kappa_4 = \mathbb{E}(X^4) - 3\sigma^4$ is the \textit{kurtosis} of $X$.
\end{itemize}
The \textbf{coefficient of skewness} $\mbox{skew}(X)$ is then $\frac{\kappa_3}{\sigma^3}$ and \textbf{coefficient of kurtosis} $\mbox{kurt}(X)$ is $\frac{\kappa_4}{\sigma^4}$.

\section*{Other Formulae}
\subsubsection*{Expected Values via. Tail Probabilities}
For a positive RV $X$, ($\mathbb{P}(X\geq 0) = 1$)$$\mathbb{E}(X^n) = n \int_0^{\infty}x^{n-1}(1-F_X(x))dx$$
\subsubsection*{Chebyshev's Inequality}
$$\mathbb{P}\left(\frac{|X-\mu|}{\sigma} \geq k\right) \leq \frac{1}{k^2}.$$
\subsubsection*{Common Series}
\begin{align*}
    &\sum_{k=0}^{\infty}\frac{x^k}{k!} &&= e^x &&\mbox{(Taylor Series)} \\
    &\sum_{i=1}^{n} a r^{i-1} &&= \frac{a(1-r^n)}{1-r}, && r \neq 1 \\
    &\sum_{k=0}^{\infty} a r^k &&= \frac{a}{1-r}, && r \in [0,1) \\
    &\sum_{n=1}^{\infty} \frac{1}{n^p} && \mbox{diverges for } p \leq 1
\end{align*}
\subsubsection*{Central Limit Theorem}
If $X_1, X_2,\dots$ are independent identically distributed random variables with $\mathbb{E}(X_i)=\mu$ and $\mbox{Var}(X_i)=\sigma^2$ and $S_n = X_1 + X_2 + \dots + X_n$ then;
$$Z_n = \frac{S_n - n\mu}{\sigma \sqrt{n}} \stackrel{d}{\rightarrow} \mbox{N}(0,1) \mbox{ as } n\rightarrow \infty.$$ Put otherwise, as $n \rightarrow \infty$, $S_n \stackrel{d}{\rightarrow} \mbox{N}(n\mu, n\sigma^2)$ or the sample mean $\overline{X} \stackrel{d}{\rightarrow} \mbox{N}(\mu, \frac{\sigma^2}{n})$.
\subsubsection*{Laplace Transform}
The Laplace transform of a RV $X$ is defined as $$L_X(t)=M_X(-t)=\mathbb{E}(e^{-tX}).$$ Then the inversion formula $$F_X(x) = \lim_{t\rightarrow \infty} \sum_{k\leq tx} \frac{(-t)^k}{k!}L_X^{(k)}(t)$$ can be used to recover the cumulative distribution of $X$.
\subsubsection*{Axioms of the Probability Function}
\begin{enumerate}
    \item $\mathbb{P}(A) \geq 0$
    \item $\mathbb{P}(\Omega) = 1$
    \item For \textit{disjoint} sets $A_i$, $\mathbb{P}(\bigcup_{i=1}^{\infty} A_i) = \sum_{i=1}^{\infty} \mathbb{P}(A_i)$.
\end{enumerate}
\subsubsection*{Solving $min$/$max$ Questions}
For for independent $U$ and $V$ and $Z = min(U, V)$:
\begin{align*}
    &\mathbb{P}(Z \leq z) &&= 1 - \mathbb{P}(Z > z) = 1 - \mathbb{P}(U > z, V > z) \\
    &   &&= 1 - \mathbb{P}(U > z)\mathbb{P}(V > z)
\end{align*}
$Z = max(U, V) \implies \mathbb{P}(Z \leq z) = \mathbb{P}(U \leq z, V \leq z)$
\subsubsection*{More Variance}
$$\mbox{Var}(X|Y=y) = E(X^2|Y=y) - E(X^2|Y=y)^2.$$
$$\mbox{Var}(X) = E(\mbox{Var}(X|Y)) + \mbox{Var}(E(X|Y)).$$

\end{document}